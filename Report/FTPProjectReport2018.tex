
\documentclass[10pt,twocolumn]{witseiepaper}
%
% All KJN's macros and goodies (some shameless borrowing from SPL)
\usepackage{KJN}
\usepackage[super]{nth}
\usepackage{subcaption}
\usepackage{listings}
\usepackage{amsmath}
\usepackage{epstopdf}
\usepackage{xcolor}
\usepackage{textcomp}
\usepackage{listings}
\usepackage{alltt}
\usepackage{matlab-prettifier}
\usepackage{graphicx}
\usepackage{changes}
\usepackage{makecell}
\usepackage{verbatim}
\usepackage{algorithm,algpseudocode}
\usepackage{balance}
\usepackage{pdfpages}
\usepackage{color} %red, green, blue, yellow, cyan, magenta, black, white
\definecolor{mygreen}{RGB}{28,172,0} % color values Red, Green, Blue
\definecolor{mylilas}{RGB}{170,55,241}
%\usepackage{flafter}

\lstset{language=Matlab, % Set colour for matlab code
	breaklines=true,%
	morekeywords={matlab2tikz},
	keywordstyle=\color{blue},%
	morekeywords=[2]{1}, keywordstyle=[2]{\color{black}},
	identifierstyle=\color{black},%
	stringstyle=\color{mylilas},
	commentstyle=\color{mygreen},%
	showstringspaces=false,%without this there will be a symbol in the places where there is a space
	numbers=left,%
	numberstyle={\tiny \color{black}},% size of the numbers
	numbersep=9pt, % this defines how far the numbers are from the text
	emph=[1]{for,end,break},emphstyle=[1]\color{red}, %some words to emphasise
	%emph=[2]{word1,word2}, emphstyle=[2]{style},    
}
%
% PDF Info
%
\ifpdf
\pdfinfo{
/Title (INSTRUCTIONS AND STYLE GUIDELINES FOR THE PREPARATION OF FINAL YEAR LABORATORY PROJECT PAPERS : 2005 VERSION)
/Author (Ken J Nixon)
/CreationDate (D:200309251200)
/ModDate (D:200510121530)
/Subject (ELEN417/455 Paper Format, 2005)
/Keywords (ELEN417, ELEN455, paper, instructions, style guidelines, laboratory project)
}
\fi

%%%%%%%%%%%%%%%%%%%%%%%%%%%%%%%%%%%%%%%%%%%%%%%%%%%%%%%%%%%%%%%%%%%%%%%%%%%%%%%
\begin{document}


\title{TITLE}

\author{Sasha Berkiwitz (818737) \& Lara Timm (704157)
\thanks{School of Electrical \& Information Engineering, University of the
Witwatersrand, Private Bag 3, 2050, Johannesburg, South Africa}
}


%%%%%%%%%%%%%%%%%%%%%%%%%%%%%%%%%%%%%%%%%%%%%%%%%%%%%%%%%%%%%%%%%%%%%%%%%%%%%%%
%
\abstract{}

\keywords{}

\maketitle
%\thispagestyle{empty}
\pagestyle{plain}
\setcounter{page}{1}


%%%%%%%%%%%%%%%%%%%%%%%%%%%%%%%%%%%%%%%%%%%%%%%%%%%%%%%%%%%%%%%%%%%%%%%%%%%%%%%
\section{INTRODUCTION}

File Transfer Protocol (FTP) is a protocol which is used  to transfer files between two hosts over a TCP or IP network~\cite{FTPbeginners}. In it's operation FTP makes use of two TCP connections, a control connection and a data connection. The control connection, used to send control information, is opened and remains open throughout the duration of the user session~\cite{topDownApproach6th}. The data connection is non-persistent; a new data connection is established for each new file transfer~\cite{topDownApproach6th}. The main objectives of FTP include the promotion of file sharing, to encourage the use of remote computers, to shield users from variations in file storage systems among hosts and to ensure reliable and efficient data transfer~\cite{rfc959}. 

%Using the python programming language as well as basic socket methods, a File Transfer Application is designed and tested. Additionally, a basic FTP User Interface (UI) is developed to interact with the remote FTP server in an intuitive way.

Detailed within the sections below are a overview of the implemented system and its command/reply messaging exchange, a working description of the system and code base,the results of system testing and a critical analysis thereof. Also included is the division of work between the project partners.

%%%%%%%%%%%%%%%%%%%%%%%%%%%%%%%%%%%%%%%%%%%%%%%%%%%%%%%%%%%%%%%%%%%%%%%%%%%%%%%
\section{SYSTEM DESCRIPTION} 

An FTP client and server pair has been designed and implemented as per RFC~959 specifications~\cite{rfc959}. By implementing the system in such a way, both the server and client are compatible standardised servers and clients. A limitation of the system is that it is not designed for any platform other than the Windows OS.

\subsection{FTP Server}

%run locally can be connnected to locally or on same NW
The FTP server is hosted locally, and can be accessed by an FTP client which is either connecting from the hosts computer, or from a computer running on the same Local Area Network (LAN) via the server's public IP address. 

%Authentication to maintain records/repos - remote repo is "./username" client does not have access to server files otherwise
In order to provide unique user experience, the server maintains a user repository, within the servers local file system, for each registered client. To gain access to their remote repository, the user must be successfully authenticated using their unique username and password combination. Clients trying to connect without a registered username and password are not permitted to access the server file system.

%uses RFC959 - can be connected to by a range of clients (standard also)
%has basic implementation plus more functionality
%server response sent in accordance w protocol
The server is implemented in accordance with the RFC~959 specification. The implemented features of the system go beyond those of the minimum requirements set out by the standard, improving the system's general functionality and usability. Following RFC~959 allows for the server to be accessed by not only the designed client but also standardised FTP clients. 

%multithreaded
To allow for more than one client to be connected to the server at a time, the server program is multithreaded. Each new client connection is handled by a separate thread, and up to five simultaneous connections can be established by FTP clients. 

%errors are communicated in messages to client
To communicate file, system and transfer status to the client, the server makes use of a number of RFC~959 specified reply codes. These codes enable the client to detect errors and react accordingly. Further discussion of this process can be found in Section~\ref{sec:command/reply}


\subsubsection*{Unimplemented features: }
In accordance with the RFC~959 minimum requirements, the default structure and file transmission mode for exchange should be implemented. Functionality for Record and Page structures, as well as for Block and Comprressed transmission modes were not implemented. The project group deemed theses features unnecessary as standard FTP clients would be able to transmit in File structure and Stream transmission modes which are default.

%recore and page structure
%block and compressed transmission modes

\subsection{FTP Client}

The FTP client is composed of two working parts, a Graphical User Interface (GUI) and a logical FTP client. The user interacts with the GUI which instructs the FTP client to interact with the FTP server. In this way the interface is separated from the logic layer, following the separation of concerns principle.

\subsubsection{Client Logic}
~\\

%runs from local pc
The FTP client is hosted locally and can connect to a server hosted either locally (on the client's computer) or over a LAN connection. To connect to the server, the client must know the public IP address and port on which the server is listening for a connection. These details are obtained from the GUI described in Section~\ref{GUI}.

%RFC - interacts w standard server
The responsibility of the FTP client is to interact with the server in a manner in which the server understands. The client translates the raw data received from the GUI by formatting it to comply with the RFC~959 specification. In this way, the client can not only interact with the designed server, but also  with a range of standard FTP servers.

%local error handling - not responsibility of server to deal w unusual requests
It is the responsibility of the client to handle errors that are not relevant to the server. Such errors include: trying to upload a file that doesn't exist in the clients working directory and handling errors to do with incorrect formatting of client commands. Although these checks do exist, the client interface will most oftem prevent these user errors from occurring.


\subsubsection{Client Interface}\label{GUI}
~\\
%The FTP client is handled by a simple GUI. 

%input params (name pass addr IP)

%can view and nav both file systems

% can upload from local or dl from remote - file saved to current working dir

%can delete files/folders (folder removes all contents) %%MAKE POPUP TO ASK IF SURE%??
%can make folders local and remote?

%client can input custom commands if necessary.

%client disconnects by?


\subsubsection*{Unimplemented features: }
In accordance with the server implementation, the client is only capable of handling the File structure and Stream mode of data transmission as specified in the minimum RFC~959 requirements.

%in accordance with server stru and mode
%cannot make new files on either side, copy paste etc. 
%cannot remane files/folders on either side
%cannot edit files on either side


%%%%%%%%%%%%%%%%%%%%%%%%%%%%%%%%%%%%%%%%%%%%%%%%%%%%%%%%%%%%%%%%%%%%%%%%%%%%%%%
\section{FTP COMMAND/REPLY OVERVIEW}\label{sec:command/reply} % talk about implemented functions and features from a server perspective - command yields reply

The format of replies to FTP commands are designed to make sure that requests and actions are well synchronized when transferring files, and to keep the client informed about the status of the server~\cite{rfc959}. There are 5 categories of FTP replies, characterised by the first digit of the three digit reply code~\cite{rfc959}. 

The categories are: 

\textbf{1**	Positive Preliminary reply:} 
requested action initiated, expect reply before sending new command.

\textbf{2**	Positive Completion reply:} 
Requested action completed, new command can be sent.

\textbf{3**   Positive Intermediate reply:} 
Command received, server waiting for further information.

\textbf{4**   Transient Negative Completion reply:} 
Command not accepted, action did not take place. Error is temporary and action may be requested again.

\textbf{5**   Permanent Negative Completion reply:} 
Command not accepted, action did not take place. User discouraged from repeating same request.

In the designed file transfer application, at least one reply was implemented from each category. A list of the client commands features and their associated server responses are detailed in Appendix~\ref{comm\reply table}

%%%%%%%%%%%%%%%%%%%%%%%%%%%%%%%%%%%%%%%%%%%%%%%%%%%%%%%%%%%%%%%%%%%%%%%%%%%%%%%
\section{DETAILED FEATURE IMPLEMENTATION} % talk about details of how the complex functions work

\subsection{Client Authentication}

\subsection{Directory Traversal}

\subsection{Data Connection Initiation}

\subsection{File Transfer}

\subsection{Miscellaneous}



%\begin{figure}[h]
%	\centering
%	\includegraphics[width=0.9\columnwidth]{collisions.png}
%	\caption{Illustration of the collision relationships that exist between game objects. Arrowheads indicate damage dealt.}
%	\raggedright
%	\label{fig:collisions}
%\end{figure}

%%%%%%%%%%%%%%%%%%%%%%%%%%%%%%%%%%%%%%%%%%%%%%%%%%%%%%%%%%%%%%%%%%%%%%%%%%%%%%%
\section{DIVISION OF WORK}

%%%%%%%%%%%%%%%%%%%%%%%%%%%%%%%%%%%%%%%%%%%%%%%%%%%%%%%%%%%%%%%%%%%%%%%%%%%%%%%
\section{RESULTS}\label{results}

%%%%%%%%%%%%%%%%%%%%%%%%%%%%%%%%%%%%%%%%%%%%%%%%%%%%%%%%%%%%%%%%%%%%%%%%%%%%%%%
\section{CRITICAL ANALYSIS}

%%%%%%%%%%%%%%%%%%%%%%%%%%%%%%%%%%%%%%%%%%%%%%%%%%%%%%%%%%%%%%%%%%%%%%%%%%%%%%%
\section{CODE STRUCTURE}

%%%%%%%%%%%%%%%%%%%%%%%%%%%%%%%%%%%%%%%%%%%%%%%%%%%%%%%%%%%%%%%%%%%%%%%%%%%%%%%
\section{CONCLUSION}


%%%%%%%%%%%%%%%%%%%%%%%%%%%%%%%%%%%%%%%%%%%%%%%%%%%%%%%%%%%%%%%%%%%%%%%%%%%%%%%
%
%\nocite{*}
\bibliographystyle{witseie}
\bibliography{FTPbib}

\newpage

\begin{appendix}
	
\section{Implemented FTP Commands/Replies}\label{comm\reply table}


\end{appendix}

%{\tiny \vfill \hfill \today \hspace{5mm} witseie-paper-2003.\TeX}


\end{document}

" vim: ts=4
" vim: tw=78
" vim: autoindent
" vim: shiftwidth=4
